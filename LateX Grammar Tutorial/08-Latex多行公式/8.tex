\documentclass{ctexart}
\usepackage{amsmath}
\usepackage{amssymb}

\begin{document}
    % 带编号
    \begin{gather}
        a+b=b+a\\
        ab  ba
    \end{gather}

    % 不带编号
    \begin{gather*}
        a+b=b+a\\
        3+5=5+3=8\\
        3\times 5=5\times 3=15
    \end{gather*}

    % 在\\前使用\notag阻止编号
    \begin{gather}
        a+b=b+a\\
        3+5=5+3=8 \notag \\
        3\times 5=5\times 3=15
    \end{gather}

    % align和align*环境(用&进行对齐)
    % 带编号
    \begin{align}
        x &= t + \cos t + 1 \\
        x &= 2 \sin t
    \end{align}
    % 不带编号
    \begin{align*}
        x &= t + \cos t + 1  & x &= \cos t & x &= t\\
        x &= 2 \sin t & y &= 2t & y &= \sin(t+1)
    \end{align*}

    % split环境(对齐采用align环境的方式,编号在中间)
    % 带编号
    \begin{equation}
        \begin{split}
            \cos 2x &= \cos ^2 x-\sin^2 x\\
                    &= 2\cos^2x -1
        \end{split}
    \end{equation}
    % 不带编号
    \begin{equation*}
        \begin{split}
            \cos 2x &= \cos ^2 x-\sin^2 x\\
                    &= 2\cos^2x -1
        \end{split}
    \end{equation*}

    % case环境
    % 每行公式使用&分割为两部分,
    % 通常表示值和后面的条件
    \begin{equation}
        D(X)=\begin{cases}
            1,& \text{如果 }x \in \mathbb{Q};\\
            0,& \text{如果}x \in \mathbb{R} \setminus\mathbb{Q}
        \end{cases}
    \end{equation}
\end{document}