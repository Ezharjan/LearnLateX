\documentclass{ctexart}
\usepackage{amsmath}
\newcommand{\adots}{\mathinner{\mkern2mu\raisebox{0.1em}{.}\mkern2mu\raisebox{0.4em}{.}\mkern2mu\raisebox{0.7em}{.}\mkern1mu}}

\begin{document}
    \textbf{matrix 无括号}
    \[
   \begin{matrix}
       0 & 1\\
       1 & 0
   \end{matrix} 
   \]

   \textbf{pmatrix 小括号}
   \[
   \begin{pmatrix}
       0 & i\\
       -i & 0
   \end{pmatrix} 
   \]

   \textbf{bmatrix 中括号}
   \[
   \begin{bmatrix}
       0 & 1\\
       1 & 0
   \end{bmatrix} 
   \]

   \textbf{Bmatrix 大括号}
   \[
   \begin{Bmatrix}
       -1 & 1\\
       1 & 1
   \end{Bmatrix} 
   \]

   \textbf{vmatrix 单竖线}
   \[
    \begin{vmatrix}
        a & b\\
        c & d
    \end{vmatrix} = ad-bc
    \]

    \textbf{Vmatrix 双竖线}
   \[
    \begin{Vmatrix}
        a & b\\
        c & d
    \end{Vmatrix} 
    \]


    \textbf{使用上下标}
    \[
   A=\begin{pmatrix}
       a_{11}^2 & a_{12}^2 & a_{13}^2\\
       0 & a_{22} & a_{23}\\
       0 & 0 & a_{33}
   \end{pmatrix} 
   \]

   \textbf{常用省略号:\textbackslash dots, \textbackslash vdots, \textbackslash ddots}
   \[
   A=\begin{bmatrix}
       a_{11} & \dots & a_{1n}\\
       \vdots & \ddots & \vdots \\
       0 & \dots & a_{nn}
   \end{bmatrix}_{n \times n}
   \]

   \textbf{自定义符号:\textbackslash adots}
   \[
   A=\begin{bmatrix}
       a_{11} & \dots & a_{1n}\\
       \vdots & \adots & \vdots \\
       0 & \dots & a_{nn}
   \end{bmatrix}_{n \times n}
   \]

   \textbf{分块矩阵}
   \[
   \begin{pmatrix}
       \begin{matrix}1&0\\0&1\end{matrix} & \text{\LARGE 0}\\
        \text{\LARGE 0} & \begin{matrix}1&0\\0&1\end{matrix}
   \end{pmatrix}
   \]

   \textbf{三角矩阵}
   \[
   \begin{pmatrix}
    a_{11} & a_{12} & \dots & a_{1n}\\
    & a_{22} &\cdots & a_{2n}\\
    &        & \ddots & \vdots \\
    \multicolumn{2}{c}{\raisebox{1.3ex}[0pt]{\Huge 0}}
    & & a_{nn}
   \end{pmatrix}_{n\times n}
   \]

   \textbf{跨列省略号:\textbackslash hdotsfor\{<列数>\}}
   \[
   \begin{pmatrix}
    1 & \frac12 & \dots & \frac1n \\
    \hdotsfor{4} \\
    m & \frac m2 & \dots & \frac mn
   \end{pmatrix}
   \]

   \textbf{使用smallmatrix环境产生行内小矩阵}

   \ \ \ \ 复数 $z=(x,y)$也可用矩阵\begin{math}
      \left(
          \begin{smallmatrix}
              x & -y\\
              y & x
          \end{smallmatrix}
      \right) 
   \end{math}来表示。

   \textbf{array环境,类似于表格环tabular}
   \[
   \begin{array}{r|r}
       \frac12 & 0 \\
       \hline 
       0 & -\frac{a}{bc}
   \end{array}
   \]

   \textbf{用array环境构造复杂矩阵}
%    @{<内容>}-添加任意内容,不占表项计数
%    此处添加一个负值空白,表示向左移5pt的距离
   \[
   \begin{array}
       {c@{\hspace{-5pt}}l}
       % 第一行第一列
       \left(
       \begin{array}{ccc|ccc}
           a & \cdots & a & b & \cdots & b \\
           & \ddots & \vdots & \vdots & \adots \\
           &        & a & b \\ \hline
           &        &   & c & \cdots & c \\
           &        &   & \vdots & & \vdots \\
           \multicolumn{3}{c|}{\raisebox{2ex}[0pt]{\Huge 0}} & c & \cdots & c
       \end{array}    
       \right)
       &
       % 第一行第二列
       \begin{array}{l}
        %\left.仅表示与\right\}配对,什么都不输出
        \left. \rule{0mm}{8mm}\right\}p\\
        \\
        \left. \rule{0mm}{8mm}\right\}q
       \end{array}
       \\[-5pt]
       % 第二行第一列
       \begin{array}{cc}
           \underbrace{\rule{15mm}{0mm}}_m & 
           \underbrace{\rule{15mm}{0mm}}_n
       \end{array}
       % 第二行第二列
   \end{array}    
   \]


\end{document}